1. What is java Script?

Ans:  JavaScript is a high-level dynamic, interpreter(line by line excution) programming language.

     it allows developers to add interactivity, dynamic effects, and 
     responsiveness to web pages.

Role of javascript in web development:

*Interactive web pages:
     it allows develpers to respond to user interaction.
     Such as-clicks, hover effets, Animation & form submissions.
   
*Dynamic content:
     java script enables developers to update web page content 
     dynamically without requiring a full page reload.

*Create web application:
     to crerate complex web application.
     such as-single page applications(SPAs),
     progressive web app (PWAS) and mobile applications.

*Dynamic function:
     we can use interchange & flexibally data type in javescript it allows same
     consideration and identify also excute which wont work like as static.

     we use this language reduce the code as much as possible than other 
     programming lang.

2. variables and Types:

Ans:
Declare a varible named userAge to numerical value

     let userAge = 25;

Declare a varible named userName to String value

     let userName = "mani";

Output both variable using console.log

           console.log ("user Age:", userAge),
           console.log("user Name:", userName),

out put = user Age: 25, user Name: mani.

3. Comments in JavaScript:

Ans:
here multi-line comment explaining the purpose of a function:

* Calculates the area of a rectangle.
 
* This function takes two parameters: length and width.
* It returns the product of these two values, representing the area of the rectangle.
 
* Parameters:
    length (number): The length of the rectangle.
    width (number): The width of the rectangle.

* Returns:
    number: The area of the rectangle.

4. Create two variables and assign them numerical values

Anss:
let num1 = 10;
let num2 = 5;

*Addition
let addition = num1 + num2;
console.log("Addition:", addition);

*Subtraction
let subtraction = num1 - num2;
console.log("Subtraction:", subtraction);

*Multiplication
let multiplication = num1 * num2;
console.log("Multiplication:", multiplication);

*Division
let division = num1 / num2;
console.log("Division:", division);

*When you run this code, it will output:


Addition: 15
Subtraction: 5
Multiplication: 50
Division: 2

5.Here's an example code that demonstrates different data types:


*String data type:

let name = "John Doe";
console.log("Name:", name);
console.log("Type:", typeof name);

*Number data type:
let age = 30;
console.log("Age:", age);
console.log("Type:", typeof age);

*Boolean data type:
let isAdmin = true;
console.log("Is Admin:", isAdmin);
console.log("Type:", typeof isAdmin);

*Array data type:
let colors = ["red", "green", "blue"];
console.log("Colors:", colors);
console.log("Type:", typeof colors);


     
String
-      Strings can contain letters, numbers, and special characters.
-      Example: "John Doe", 'Hello World'

Number
-      A number is a numeric value that can be an integer or a floating-point number.
-      Numbers can be positive or negative.
-      Example: 30, -10.5

Boolean
-      A boolean is a logical value that can be either true or false.
-      Booleans are often used in conditional statements and loops.
-      Example: true, false

Array
-      An array is a collection of values that can be of any data type, including strings, numbers, booleans, and other arrays.
-      Arrays are enclosed in square brackets [] and elements are separated by commas.
-      Example: ["red", "green", "blue"], [1, 2, 3]


6.Functions in JavaScript:

Write a function named greetUser that takes a name parameter and returns a greeting message.
Call the function with a sample name and display the result using console.log()?

Ans:
     
function greetUser(name){
     return "hello, "+name
 }
 
 let nameUser = "mani"
 let myname =greetUser(nameUser)
 
 console.log(myname)

      output will come: hello, mani

7. if Else in JavaScript:

 variable named temperature and assign it a numerical value.

here an if-else statement to check if the temperature is greater than 30:


var temperature =31

if(temperature >30)
{console.log("Heavy sun light")}

else if (temperature<=30)
{console.log("sun is pleasant")}

output will :Heavy sun light



8.FOR LOOP:

Use a for loop to print the numbers from 1 to 5 in the console.


* Basic for loop: prints numbers 1 to 5

for (let i = 1; i <= 5; i++) {
    console.log(i);
  }

 9. Loose vs Strict Equality:

Explain the difference between loose equality (==) and strict equality (===) with examples.

Loose Equality (==) Examples:

console.log("0" == false);  // Output: true
console.log(null == 0);  // Output: false
console.log(undefined == 0);  // Output: false
console.log(" \t\r\n" == 0);  // Output: true


Strict Equality (===) Examples:

console.log("0" === false);  // Output: false
console.log(null === 0);  // Output: false
console.log(undefined === 0);  // Output: false
console.log(" \t\r\n" === 0);  // Output: false


What happens during Loose Equality (==) checks?
  Type Conversion: JavaScript attempts to convert one or both operands to a common type.
  Value Comparison: After type conversion, JavaScript compares the values.

Pitfalls of Loose Equality (==)
  Unexpected Type Conversions: Can lead to incorrect results.
  Null and Undefined: Can be equal to other values, causing unexpected behavior.

Benefits of Strict Equality (===)
  Predictable Behavior: Ensures accurate comparisons without unexpected type conversions.
  Code Reliability: Helps prevent bugs and errors caused by loose equality checks.

In summary:

- Loose equality (==) performs implicit type conversions, which can lead to unexpected results.
- Strict equality (===) checks both value and data type, ensuring more accurate comparisons.
- Use strict equality (===) for most comparisons to avoid unexpected type conversions.








    

     


          