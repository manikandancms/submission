HTML
1. WHAT IS HTML
Ans: HTML Stand for = (Hyper Text Markup Language)
2. WHY HTML IS USED
Ans: HTML is used to structure and define the content of web pages.
3. What is Emmet & List 5 Emmet
Ans: Emmet is a popular plugin for text editors that allows you to write HTML and CSS code faster and more efficiently.
(i) a
<a href=""></a>
 (ii) !
Alias of html:5
<!DOCTYPE html>
<html lang="en">
<head>
    <meta charset="UTF-8" />
    <title>Document</title>
</head>
<body>
</body>
</html>
(iii) br
<br />
(iv) ul>li*2
<ul>
    <li></li>
(v) a:link
<a href="http://"></a>

4. SHORT CUT CREATE of Emmet Boiler plate of HTML
Ans: To create a basic HTML boilerplate using Emmet, you can use the following shortcut:
*html:5*

<!DOCTYPE html>
<html >
<head>
  <title>Document</title>
</head>
<body>
</body>
</html>

 5. WHAT STRUCTURE OF HTML
Ans: Here is the basic structure of an HTML document:
<!DOCTYPE html>
<html >
<head>
    <title>Document</title>
</head>
<body>  
</body>
</html>

6.  WHAT IS **HEAD** TAG
Ans: The <head> # </head> tag plays a crucial role in:
	Links: Link to external stylesheets, scripts, or other resources.
	Scripts: Include JavaScript code or link to external scripts.
	Styles: Define internal styles or link to external stylesheets.
	Tailwind: To Add the tailwind script

 7. WHAT IS BODY TAG
Ans: The <body> tag serves as a container for all the visible content of a web page, including:
•	Text: Paragraphs, headings, lists
•	Images: Graphics, logos, icons 
•	Multimedia: Audio, video
•	Interactive elements: Forms, buttons, links
•	Semantic elements: Structural elements like header, footer, nav 

8.  IMPORTANT 3 ELEMENTS OR TAG
Ans: *Headings (<h1>, <h2>, <h3>, etc.)*= Headings create a clear visual hierarchy
*Links (<a>) and Anchors* = Links enable users to navigate between pages and sections, making it essential for website usability.

*Semantic Containers*(<header>, <nav>, <main>, <section>, <footer>)*= Semantic containers provide a clear structure for screen readers, making it easier for users with disabilities to navigate the page.


9.  WHAT IS DIFFERENT B/W ELEMENTS & TAGS IN HTML
Ans: Tag are using opening element and closing element = example:  <p> </p>
Element has content between opening tag and closing tag like full form = example: <p> I’m very happy to see you </p>



CSS

1.	WHAT IS CSS
Ans: CSS stands for Cascading Style Sheets.
2.	WHY IS CSS
 Ans: It is essential for web development because it controls the presentation of web pages.

3.	HOW TO CONNECT HTML & CSS , Tell the emmet Shortcut

Ans:
 <head>
  <link rel="stylesheet" type="text/css" href="../style.css">
</head>

Emmet: “link:css”
4.	HOW TO GET FONT & COLOR , IMAGE, CDN
Ans:
 Font  
Go to google font
•  Choose your desired fonts. 
•  Copy the provided <link> tag and paste it into the <head> section of your HTML document. 
•  Use the corresponding CSS font family property to apply the font to your elements.

CSS Color Values: 
•	CSS supports various color formats, including: 
o	Hexadecimal (e.g. #FF000 for red).
o	RGB (e.g. rcg (255, 0, 0)
o	
•	Image Formats: 
o	Use appropriate image formats: 
	JPEG: For photos.   
	PNG: For images with transparency.   
	WebP: A modern format that provides excellent compression.
CDNs (Content Delivery Networks):
•	Purpose: 
o	CDNs distribute your website's static files (images, CSS, JavaScript) across a network of servers worldwide. This improves loading speed by serving files from the server closest to the user
o	Ex: bootstrap, imagekit
5.	BASIC CSS FAMILY : Text Family, Color & Background Family, Box Model
Ans: 
Text Family:
This family deals with the styling of text. Key properties include:
Font family
o	Specifies the typeface to be used.
o	Examples: font -family: Arial, sans sarif.
o	Sets the size of the text.
o	Examples: font-size: 16px;, font-size: 1.2em;
	font-weight:
o	Controls the boldness of the text.
o	Examples: font-weight: bold;, font-weight: 400;
font-style:
o	Sets the style of the text (e.g., italic).
o	Example: italic
font-style:
o	Sets the color of the text.
Example:blue
Color & Background Family:
This family focuses on the colors and backgrounds of elements. Key properties include:
Color:
o	Sets the foreground color of an element (typically text).
Backgroundcolor:
o	Sets the background color of an element.
o	Example: ackground-color: lightgray;
Box Model:
The CSS box model is fundamental to understanding how elements are laid out on a webpage. It consists of:
•	Content: 
o	The actual content of the element (text, images, etc.).
•	Padding: 
o	The space between the content and the border.
	Ex:padding 10 px

Border: 
A line that surrounds the padding and content.
                  Ex: boder boder solid

•	Margin: 
o	The space outside the border, separating the element from other elements.
	Ex: margin 5px

6.	WHEN TO USE BACKGROUND IMG VS IMG TAG
Ans: 
*Background Images (background-image)*
1. Use for decorative purposes: Background images are suitable for decorative elements, such as textures, patterns, or gradients, that don't add essential meaning to the content.
2. Use for responsive designs: Background images can be easily resized and adapted to different screen sizes and devices using CSS.
3. Use for visual effects: Background images can be used to create visual effects, such as parallax scrolling or hover effects.

*Img Tag (<img>)*
1. Use for content images: The img tag is suitable for images that add essential meaning to the content, such as product images, infographics, or photos.
2. Use for semantic meaning: The img tag provides semantic meaning to the image, making it accessible to screen readers and search engines.
3. Use for image captions: The img tag allows you to add captions to images using the figcaption element.
7.   WHAT IS LAYOUT & Explain About flexbox Properties
Ans:
	display: flex or inline-flex - Defines the container as a flexible box.
	flex-direction: row, row-reverse, column, or column-reverse - Specifies the direction of the flexible items.
	flex-wrap: wrap, nowrap, or wrap-reverse - Controls whether the flexible items wrap to a new line or not.
	justify-content: flex-start, flex-end, center, space-between, or space-around - Aligns the flexible items along the main axis.
	align-items: flex-start, flex-end, center, baseline, or stretch - Aligns the flexible items along the cross axis.
	align-content: flex-start, flex-end, center, space-between, space-around, or stretch - Aligns the flexible lines along the cross axis.
	flex-grow: <number> - Specifies the growth factor of a flexible item.
	flex-shrink: <number> - Specifies the shrink factor of a flexible item.
	flex-basis: <length> or <percentage> - Specifies the initial length of a flexible item.

•	Flexbox Container Properties:
1.	flex: shorthand for flex-grow, flex-shrink, and flex-basis.
2.	flex-flow: shorthand for flex-direction and flex-wrap.

•	Flexbox Item Properties:
1.	align-self: auto, flex-start, flex-end, center, baseline, or stretch - Aligns the item along the cross axis.
2.	flex: shorthand for flex-grow, flex-shrink, and flex-basis.
3.	order: <integer> - Specifies the order of the item.


8.WHAT IS TAILWIND CSS & WHY IS IT USED

•	What is Tailwind CSS?
•	Tailwind CSS is a utility-first CSS framework that allows you to write more concise and maintainable CSS code. It provides a set of pre-defined classes that can be used to style HTML elements.

•	Key Features:
1.	Utility-first approach: Tailwind CSS focuses on providing low-level utility classes that can be combined to create custom styles.
2.	Configurable: Tailwind CSS allows you to customize the framework by configuring the settings in the tailwind.config.js file.
3.	Responsive design: Tailwind CSS provides a set of responsive design classes that make it easy to create mobile-friendly and responsive designs.
4.	Pre-defined classes: Tailwind CSS provides a set of pre-defined classes for common styling tasks, such as spacing, sizing, and typography.

•	Why is Tailwind CSS used?
1.	Faster development: Tailwind CSS speeds up development by providing pre-defined classes that can be used to style HTML elements quickly.
2.	More maintainable code: Tailwind CSS encourages a utility-first approach, which results in more maintainable and modular code.
3.	Customizable: Tailwind CSS allows developers to customize the framework to fit their specific needs.
4.	Responsive design: Tailwind CSS makes it easy to create responsive designs that work well on different devices and screen sizes.
5.	Large community: Tailwind CSS has a large and active community, which means there are many resources available for learning and troubleshooting.








  





